%%%%%%%%%%%%%%%%%%%%%%%%%%%
\chapter {Introduction}
\label{INTRO}
%%%%%%%%%%%%%%%%%%%%%%%%%%%
\noindent Location-privacy is becoming a major concern in the Opportunistic Mobile Social Network (OMSN) which is a kind of Delay Tolerant Networks (DTNs) \cite {C1} featuring lack of continuous connectivity. More specifically, in OMSNs, it is not necessary for senders to have an end-to-end routing path to their destinations. Users make contacts when they encounter each other. LBSs are common applications in OMSNs and they are widely used in "military and government industries, emergency services and the commercial sector" \cite {C4}, especially after the proliferation of localization technologies, like GPS. Many people access to LBSs with their portable devices and send their location to LBS providers, which expose their trace in front of the LBS. In this case, LBS users face a continuous risk that their location may be leaked from LBS applications. That makes people unwilling to use LBSs. Thus, protect location privacy has been a critical issue in LBSs.


\subsection{ Motivation}

\noindent Location-Based Services (LBSs) which use the location information of users can be considered as ?two types of application design: push and pull services? \cite {C4}. For example, you may receive an advertisement when you enter an area, which is a push service; if you look for the nearest restaurant, you must pull information from the network. It is obvious that you must expose a location if you use LBS. When you use a LBS application to find the nearest restaurant, you actually tell the LBS provider your location and your next destination which might be a restaurant nearby. If attackers can access the LBS provider's database, they can learn that information. Then you may receive a lot of advertisements from surrounding restaurants.

The above inference attack just bothers you with advertisements, but it becomes more dangerous when you send more queries to the LBS provider. If you use the former LBS application 10 times in a day, attackers can learn your trace from the information so that they can infer more information including your identity and home address. ?Hoh used a database of week-long GPS traces from 239 drivers in the Detroit, MI area. Examining a subset of 65 drivers, their home-finding algorithm was able to find plausible home locations of about 85\%, although the authors did not know the actual locations of the drivers' homes? \cite {C5}. Therefore, the LBS provider is considered as a semi-trusted party so that users must protect their location-privacy when they use LBS applications.

The OMSN is defined ``as decentralized opportunistic communication networks formed among human carried mobile devices that take advantage of mobility and social networks to create new opportunities for exchanging information and mobile ad hoc social networking'' \cite {C2}. In other words, OMSN is a kind of mobile ad hoc network, and the information and technology of social network are also used in it. People carrying smartphones which contains WiFi or Bluetooth equipments can compose a typical OMSN. Since the WiFi, Bluetooth equipment on smartphones can be used for discovering other devices and direct communication between devices, users can communicate with others who are in their communication ranges without any infrastructure, which is the basic requirement of an OMSN. Since users in OMSN could also use LBS applications, their location-privacy must be protected. Besides, the information in the social network allows users to identify friends. As a result, we can design a location-privacy protection protocol using the social network.


\subsection{ Objectives}

\noindent Disconnection, decentralization and highly delays are obvious features for OMSN so that a decentralized strategy may benefit for the location-privacy protection protocol in OMSN. Besides, decreasing interactions between users and servers can significantly cut down the time cost. To prevent attackers from learning users' private information, the protocol should obfuscate users' location and hide their identities.


\subsection{ Contribution}

\noindent We propose two new decentralized location-privacy protecting protocols for OMSN so that they do not need a three-party server which is used to obfuscate queries for users. We also create a new simulator for OMSNs called OMSN Routing protocols Simulator (ORS) to evaluate our protocols.

The first protocol which we propose is a distributed location-privacy algorithm, called Multi-Hop Location-Privacy Protection (MHLPP). It guarantees location-privacy and achieves a higher query success ratio. The introduction of social networks enables us to hide the original requester's information behind his friends. When a user wants to send a query, he starts to look for friends based on information in his social network. He sends his query to the first encountered friend who is then responsible for forwarding the query to the intended location. This friend can also pass the query to one of his friends when they encounter. When the distance between the user carrying this query and the original requester exceeds a specified threshold, the user sends the query to the LBS server directly without having to find a friend to pass on. At that time, he also replaces the original requester's information with its own identity and location, which enables the LBS server to receive the query without any information about the original requester. After receiving the query, the LBS server replies to the last friend (the user sending the query to the LBS) who then transmits it to the original requester.

Our second protocol is also a distributed location-privacy algorithm called Appointment Card Protocol (ACP), which aims to guarantee location-privacy and reach a higher query success ratio. The introduction of social networks enables us to hide the original requester's information behind his friends. We introduce the Appointment Card (AC) as a kind of intermediary which records a serial of agents. The original requester sends his query using the identity of the first agent in the AC to the LBS, which prevent the LBS learns the identity of the requester. The reply of the query can be delivered back to the original requester along with the same serial of agents as in the AC one by one. The last agent called the trusted agent is a user in the social network of the original requester, in other words, he is a friend (or a friend of friends) of the original requester. The trusted agent separates the stranger-agents and the original requester so that no stranger knows the identity of the original requester. The query-delivery success ratio of the ACP is in a similar level as the no-privacy protocol comparison in our experiment.

Our new simulator ORS is inspired by a well-known simulator the ONE \cite {C35}, which is used by a lot of researchers to test their models and protocols in DTN. The major reason why we create a new simulator is that the ONE is not designed for OMSN so that there is no social network context in it. Adding social network information into the ONE simulator must modify the basic structure of it, which might import risks for the correctness of our experiments. We also use more reliable and efficient algorithm in our simulator, so that our simulator is about 2 times faster than the ONE in the case where we test our protocols.


\subsection{ Thesis Organization}

\noindent This thesis is organized as follows:

Chapter 2 describes the concept of Mobile Ad hoc networks, Delay Tolerant Network, Location-Based Services and social networks. We give an overview of the existing location-privacy protocols.

Chapter 3 describes our first protocol MHLPP, which is explained in detail through architectural and mathematical instances. The comparison is made with a similar location-privacy protocol.

Chapter 4 elaborates on the second protocol ACP. We introduce a description of how it works and how it enhances the location-privacy. We also use an example to show the whole process.

Chapter 5 shows our new simulator. Also, we introduce the algorithms which are used in the simulator.




\begin{comment}


\begin{figure}[H]
  \centering  
  \includegraphics[width=1\textwidth]{figures/map2.pdf}
  \caption{A map showing the organization of the thesis}
\end{figure}

\end{comment}
