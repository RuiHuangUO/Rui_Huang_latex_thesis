%%%%%%%%%%%%%%%%%%%%%%%%%%%
\chapter {Introduction}
\label{INTRO}
%%%%%%%%%%%%%%%%%%%%%%%%%%%
Location-privacy is becoming a major concern in the Opportunistic Mobile Social Network (OMSN, which is a kind of a Delay Tolerant Network (DTN) \cite {C1} featuring lack of continuous connectivity. More specifically, in OMSNs, it is not necessary for senders to have an end-to-end routing path to their destinations. Users make contact when they encounter each other. LBSs are common applications in OMSNs and they are widely used in ``military and government industries, emergency services and the commercial sector'' \cite{C4}, especially after the proliferation of localization technologies, like GPS. Many people access LBSs with their portable devices and send their location to LBS providers. In this case, LBS users face a continuous risk that their location may be leaked from the LBS applications, which makes people unwilling to use LBSs. Thus, protecting location privacy has been a critical issue in LBSs.


\section{Motivation and Objective}
LBSs which use the location information of users can be considered as ``two types of application design: push and pull services'' \cite {C4}. For example, you may receive an advertisement when you enter an area, which is a push service; if you look for the nearest restaurant, you must pull information from the network. It is obvious that you must reveal your location if you use LBS. When you use a LBS application to find the nearest restaurant, you actually tell the LBS provider your location and your next destination which might have a restaurant nearby. If attackers can access the LBS provider's database, they can learn that information. Then you may receive a lot of advertisements from surrounding restaurants.

Such inference attack, in addition to bothering you with advertisements, can become more dangerous when you send more queries to the LBS provider. If you use an LBS application several times in a day, the attackers can learn your trace from the information so that they can infer more information including your identity and home address. According to \cite {C5}, ``Hoh used a database of week-long GPS traces from 239 drivers in the Detroit, MI area. Examining a subset of 65 drivers, their home-finding algorithm was able to find plausible home locations of about 85\%, although the authors did not know the actual locations of the drivers' homes''. Therefore, the LBS provider is considered as a semi-trusted party so that users must protect their location-privacy when they use LBS applications.

The OMSN is defined ``as decentralized opportunistic communication networks formed among human carried mobile devices that take advantage of mobility and social networks to create new opportunities for exchanging information and mobile ad hoc social networking'' \cite {C2}. In other words, OMSN is a kind of mobile ad hoc network, and the information and technology of social network are also used in it. People carrying smartphones which contain WiFi or Bluetooth can form a typical OMSN. Since the WiFi, Bluetooth on smartphones can be used for discovering other devices and direct communication between devices, users can communicate with others who are within their communication range without using any infrastructure, which is the basic requirement of an OMSN. Since users in OMSN could also use LBS applications, their location-privacy must be protected. Besides, the information in the social network allows users to identify friends. Therefore, it is necessary to have location-privacy protection protocol using the social networks.


\section{Objectives}

Disconnection, decentralization and highly delays are obvious features for OMSN so that a decentralized strategy may benefit for the location-privacy protection protocol in OMSN. Besides, decreasing interactions between users and servers can significantly cut down the time cost. To prevent the attackers from learning users' private information, the protocol should obfuscate users' location and hide their identities.


\section{Contributions}

We propose two new decentralized location-privacy protecting protocols for OMSN so that they do not need a three-party server which is used to obfuscate queries for users. We also create a new simulator for OMSNs called OMSN Routing protocols Simulator (ORS) to evaluate our protocols.

The first protocol we propose is a distributed location-privacy algorithm called Multi-Hop Location-Privacy Protection (MHLPP), which achieves a higher query success ratio and guarantees location-privacy. The introduction of social networks enables us to hide the original requester's information behind his friends. When a user wants to send a query, he starts to look for friends based on information in his social network. He sends his query to the first encountered friend who is then responsible for forwarding the query to the intended location. This friend can also pass the query to one of his friends when they encounter. When the distance between the user carrying this query and the original requester exceeds a specified threshold, the user sends the query to the LBS server directly without having to find a friend to pass on. At that time, he also replaces the original requester's information with its own identity and location, which enables the LBS server to receive the query without any information about the original requester. After receiving the query, the LBS server replies to the last friend (the user sending the query to the LBS) who then transmits it to the original requester.

Our second protocol is also a distributed location-privacy algorithm called Appointment Card Protocol (ACP), which aims to guarantee location-privacy and reach a higher query success ratio. The introduction of social networks enables us to hide the original requester's information behind his friends. We introduce the Appointment Card (AC) as a kind of intermediary which records a series of agents. The original requester sends his query using the identity of the first agent in the AC to the LBS, which prevents the LBS from learning the identity of the requester. The reply of the query can be delivered back to the original requester through the same series of agents as in the AC. The last agent, called the trusted agent, is a user of the original requester, in the social network; in other words, he is a friend (or a friend of friends) of the original requester. The trusted agent separates the stranger-agents from the original requester so that no stranger knows the identity of the original requester. The query-delivery success ratio of our ACP is as good as that of any no-privacy protocol used in comparison in our experiment.

Our new simulator ORS is inspired by a well-known simulator, ONE \cite {C35}, which is used by many researchers to test their models and protocols in DTN. The major reason why we created a new simulator is that the ONE is not designed for OMSN and lacks social networking concepts for it to be effective in OMSN. Adding social network information into the ONE simulator must modify its basic structure, which is problematic and might not ensure correctness of our experiments. We used more efficient algorithms to make our simulator run two times faster than ONE.


\section{Thesis Organization}

\noindent This thesis is organized as follows:

Chapter 2 describes the concept of Mobile Ad hoc networks, Delay Tolerant Networks, Location-Based Services and social networks. We also give an overview of the existing location-privacy protocols.

Chapter 3 describes our first protocol, MHLPP, in detail. The performance of MHLPP has been compared against its counterpart, Hybrid and Social-aware Location-Privacy in Opportunistic mobile social networks (HSLPO) \cite {C17}.

Chapter 4 elaborates on the second protocol, ACP. We describe how it works and how it enhances the location-privacy. Its performance is compared against Binary Spray and Wait (BSW) \cite{C31}, distributed social based location privacy protocol (SLPD) \cite{C16}, and our Multi-Hop Location-Privacy Protection (MHLPP).

Chapter 5 describes our new simulator which has been instrumental in simulating MHLPP and ACP in realistic settings. We introduce the key enhancements and the algorithms which are used in the simulator for making it more efficient.



\begin{comment}


\begin{figure}[H]
  \centering  
  \includegraphics[width=1\textwidth]{figures/map2.pdf}
  \caption{A map showing the organization of the thesis}
\end{figure}

\end{comment}
