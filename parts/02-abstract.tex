% A B S T R A C T
% ---------------

\begin{center}\textbf{Abstract}\end{center}

\noindent Users face location-privacy risks when accessing Location-Based Services (LBSs) in an Opportunistic Mobile Social Networks (OMSNs). In order to protect the original requester's identity and location, we propose two location privacy obfuscation protocols utilizing social ties between users.

The first one is called Multi-Hop Location-Privacy Protection (MHLPP) protocol. To increase chances of completing obfuscation operations, users detect and make contacts with one-hop or multi-hop neighbor friends in social networks. Encrypted obfuscation queries avoid users learning important information especially the original requester's identity and location except for trusted users. Simulation results show that our protocol can give a higher query success ratio compared to its existing counterpart. 

Another protocol is called Appointment Card protocol (ACP). To facilitate the obfuscation operations of queries, we introduce messages called the Appointment Card (AC). The original requesters can send their queries to the LBS directly using the information in the AC so that the query time cost of ACP is similar as no-privacy protocols ensuring that the original requester is not detected by the LBS. Also, a path for reply message is built when the query is sent, thus saving lots of time in replying queries. Simulation results show that our protocol can reach a higher query success ratio comparing to the existing protocol.

We also create a new OMSN simulator, called OMSN Routing Simulator (ORS), involving social network to provide a better evaluation of OMSN protocols. It shows more efficient and reliable performance in our experiments.


\cleardoublepage
%\newpage
