% A B S T R A C T
% ---------------

\begin{center}\textbf{Abstract}\end{center}

\noindent Users face location-privacy risks when accessing Location-Based Services (LBSs) in an Opportunistic Mobile Social Networks (OMSNs). In order to protect the original requester's identity and location, we propose two location privacy obfuscation protocols utilizing social ties between users.

The first one is called Multi-Hop Location-Privacy Protection (MHLPP) protocol. To increase chances of completing obfuscation operations, users detect and make contacts with one-hop or multi-hop neighbor friends in social networks. Encrypted obfuscation queries avoid users learning important information especially the original requester's identity and location except for trusted users. Simulation results show that our protocol can give a higher query success ratio compared to its existing counterpart. 

The second protocol is called Appointment Card Protocol (ACP). To facilitate the obfuscation operations of queries, we introduce the concept called Appointment Card (AC). The original requesters can send their queries to the LBS directly using the information in the AC, ensuring that the original requester is not detected by the LBS. Also, a path for reply message is kept when the query is sent, to help reduce time for replying queries. Simulation results show that our protocol preserves location privacy and has a higher query success ratio than its counterparts.

We have also developed a new OMSN simulator, called OMSN Routing Simulator (ORS), for simulating OMSN protocols more efficiently and effectively for reliable performance.

\cleardoublepage
%\newpage
