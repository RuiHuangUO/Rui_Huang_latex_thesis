


%%%%%%%%%%%%%%%%%%%%%%%%%%%
\chapter{ Conclusion and Future Work}
\label{CFW}
%%%%%%%%%%%%%%%%%%%%%%%%%%%
 

\noindent In this chapter, we will summarize our findings and discuss the future directions of the proposed work in the previous chapters yet to be explored. 


\section{ Summary of Work Done}

\noindent In this thesis, we proposed and analyzed two new location-privacy preserving protocols, namely MSLPP and ACP, besides developing a new OMSN simulator, called ORS, as a platform for evaluating the proposed protocols and comparing them with their counterparts.

Our first protocol, MSLPP, is a distributed location-privacy preserving protocol using social-relationship and encryption. Simulation results show that it has a better performance on delivery success ratio and provides an acceptable obfuscation compared to its counterpart HSLPO which is an improvement of their earlier scheme SLPD. Compared to HSLPO, our scheme MHLPP achieves an in increase in success ratio between 6\% and 11\% for the same range of parameters considered by the authors of SLPD. The success ratios of both HSLPO and MHLPP decline when the privacy threshold is increased, while the success ratio of HSLPO is always lower than that of MHLPP and decreases more sharply than that of MHLPP. Although HSLPO is slightly more secure than MHLPP based on our experiments which means the original requester has a lower rate to be located, but the entropy difference is quite small and negligible. Encryption and decryption to enhance security are added cost for MHLPP which is acceptable. Our experiment (using the map of Helsinki) presents that in a social network with 126 users, the average number of times of a query is encrypted and decrypted is less than 3 when the privacy threshold is considered as 85.

Our second proposed Appointment Card Protocol ACP also uses social relationship for preserving location privacy. It facilitates the obfuscation process by continuously exchanging appointment cards among users so that the original requester does not communicate with any of his friends when initiating a query. Simulation results show that it has a better performance with respect to the query-delivery success ratio and provides an acceptable obfuscation compared to its counterparts. Although ACP requires slightly more time to forward the reply, the total time required for query delivery and reply delivery is still less than that for its counterparts SLPD and MHLPP. The major disadvantage is that the ACP must exchange ACs continuously, which can consume network resources. However, that cost is low if users do not send too many queries.

\noindent To facilitate our experimentation, we have developed a simulator, called ORS, which is more suitable for OMSN simulation than the existing simulator, ONE. The ORS is more reliable and efficient than ONE for simulating opportunistic social networks. 


\section{ Future Work}

\noindent In MHLPP, we just considered the relationship strength between two connected people in the social network; in other words, an agent only gives the message to his direct friend. In most of the cases, the friends of your friends can also be trusted and be considered. If these friends can participate in forwarding queries, agents can have more choice in the obfuscation phase.

In ACP, the sizes of agents' relay tables are significantly affected by the number of appointment cards passing by them. If we can decrease the number of appointment cards in the network, the burden of agents can also decline. For example, a single appointment card can be shared among friends instead of being used by only one user.